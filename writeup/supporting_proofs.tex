\documentclass{article}

\usepackage{caption}
\usepackage{subcaption}
\usepackage{amsmath}
\usepackage{hyperref}
\usepackage{cleveref}
\usepackage[toc,page]{appendix}
\usepackage{titlesec}
\usepackage{algorithm}
\usepackage{algpseudocode}
\usepackage{soul}
\usepackage{cancel}

\usepackage{tikz}
\usetikzlibrary{arrows}
\usetikzlibrary{calc}

% General defintions
\DeclareMathOperator*{\argmax}{arg\,max}
\DeclareMathOperator*{\argmin}{arg\,min}

\newcommand{\bm}[1]{\mathbf{#1}}
\newcommand{\bs}[1]{\boldsymbol{#1}}
\newcommand{\dint}{\text{d}}

% Bold states and actions
\newcommand{\s}{\bm{s}}
\newcommand{\ac}{\bm{a}}
\newcommand{\ns}{\bm{s}'}

% Expectations and variances
\DeclareMathOperator{\E}{\mathbb{E}}
\DeclareMathOperator{\Var}{\text{Var}}
\DeclareMathOperator{\Cov}{\text{Cov}}

% Brackets
\newcommand{\rdb}[1]{\left( #1 \right)}
\newcommand{\sqb}[1]{\left[ #1 \right]}

% Caligraphics
\newcommand{\mc}[1]{\mathcal{#1}}
\newcommand{\data}{\mathcal{D}}
\newcommand{\mcs}{\mathcal{S}}
\newcommand{\mca}{\mathcal{A}}

\newcommand{\mcn}{\mathcal{N}}
\newcommand{\mct}{\mathcal{T}}
\newcommand{\mcr}{\mathcal{R}}

\newcommand{\mcz}{\mathcal{Z}}
\newcommand{\mczo}{\mathcal{Z}^*}
\newcommand{\mczpi}{\mathcal{Z}^\pi}

\newcommand{\mcw}{\mathcal{W}}
\newcommand{\mcwo}{\mathcal{W}^*}
\newcommand{\mcwpi}{\mathcal{W}^\pi}


% Bellman operators
\newcommand{\Bv}{\mathcal{B}^\pi_V}
\newcommand{\Bq}{\mathcal{B}^\pi_Q}
\newcommand{\Boptv}{\mathcal{B}^*_V}
\newcommand{\Boptq}{\mathcal{B}^*_Q}

\newcommand{\Ubell}{\mathcal{U}^\pi_t}

% Theta parameters
\newcommand{\thetamct}{\bs{\theta}_{\mathcal{T}}}
\newcommand{\thetamcr}{\bs{\theta}_{\mathcal{R}}}

\newcommand{\thetamcz}{\bs{\theta}_{\mathcal{Z}}}
\newcommand{\thetamczo}{\bs{\theta}_{\mathcal{Z}^*}}
\newcommand{\thetamczpi}{\bs{\theta}_{\mathcal{Z}^\pi}}

\newcommand{\thetamcw}{\bs{\theta}_{\mathcal{W}}}
\newcommand{\thetamcwo}{\bs{\theta}_{\mathcal{W}^*}}
\newcommand{\thetamcwpi}{\bs{\theta}_{\mathcal{W}^\pi}}
\newcommand{\btheta}{\boldsymbol{\theta}}

% V and Q
\newcommand{\V}{V^{\pi}}
\newcommand{\Q}{Q^{\pi}}
\newcommand{\Vo}{V^*}
\newcommand{\Qo}{Q^*}
\newcommand{\Vosbu}{V^*_{\mathbf{s}}}
\newcommand{\Vonbu}{V^*_{\mathbf{s}'}}
\newcommand{\Qosbu}{Q^*_{\mathbf{s}, \mathbf{a}}}
\newcommand{\Qonbu}{Q^*_{\mathbf{s}', \mathbf{a}'}}

\newcommand{\Vsbu}{V^{\pi}_{\mathbf{s}}}
\newcommand{\Vnbu}{V^{\pi}_{\mathbf{s}'}}
\newcommand{\Qsabu}{Q^{\pi}_{\mathbf{s}, \mathbf{a}}}
\newcommand{\Qsanbu}{Q^{\pi}_{\mathbf{s}', \mathbf{a}'}}

\newcommand{\Qsatbu}{Q^{\pi}_{\mathbf{s}, \mathbf{a}, t}}
\newcommand{\Qsantbu}{Q^{\pi}_{\mathbf{s}', \mathbf{a}', t+1}}

\newcommand{\Qsabut}{Q_{\mathbf{s}_t, \mathbf{a}_t}}
\newcommand{\Qsanbut}{Q_{\mathbf{s}_{t+1}, \mathbf{a}_{t+1}}}

% Rewards
\newcommand{\rsan}{r_{s,a,s'}}
\newcommand{\rsanb}{r_{\bm{s}, \bm{a}, \bm{s}'}}
\newcommand{\rsanbt}{r_{\bm{s}, \bm{a}, \bm{s}', t}}
\newcommand{\rbarsanb}{\bar{r}_{\bm{s}, \bm{a}, \bm{s}'}}

% W
\newcommand{\wb}{w^\pi_{\bm{s}}}
\newcommand{\wnb}{w^\pi_{\bm{s}'}}

\newcommand{\wob}{w^*_{\bm{s}}}
\newcommand{\wonb}{w^*_{\bm{s}'}}

% Z
\newcommand{\zb}{z^\pi_{\bm{s}, \bm{a}}}
\newcommand{\znb}{z^\pi_{\bm{s}', \bm{a}'}}

\newcommand{\zob}{z^*_{\bm{s}, \bm{a}}}
\newcommand{\zonb}{z^*_{\bm{s}', \bm{a}'}}

% BQL
\newcommand{\bqlmu}{\mu_{\zob}}
\newcommand{\bqltau}{\tau_{\zob}}

\newcommand{\bqlmuzero}{\mu_{0_{\mathbf{s}, \mathbf{a}}}}
\newcommand{\bqllambda}{\lambda_{\mathbf{s}, \mathbf{a}}}
\newcommand{\bqlalpha}{\alpha_{\mathbf{s}, \mathbf{a}}}
\newcommand{\bqlbeta}{\beta_{\mathbf{s}, \mathbf{a}}}

% PSRL
\newcommand{\thetamcthat}{\boldsymbol{\hat{\theta}}_{\mct}}
\newcommand{\thetamcrhat}{\boldsymbol{\hat{\theta}}_{\mcr}}

\newcommand{\thetamctpost}{p(\thetamct | \data)}
\newcommand{\thetamcrpost}{p(\thetamcr | \data)}




% if you need to pass options to natbib, use, e.g.:
%     \PassOptionsToPackage{numbers, compress}{natbib}
% before loading neurips_2019

% ready for submission
% \usepackage{neurips_2019}

% to compile a preprint version, e.g., for submission to arXiv, add add the
% [preprint] option:
%     \usepackage[preprint]{neurips_2019}

% to compile a camera-ready version, add the [final] option, e.g.:
     \usepackage{neurips_2019}

% to avoid loading the natbib package, add option nonatbib:
%     \usepackage[nonatbib]{neurips_2019}

\usepackage[utf8]{inputenc} % allow utf-8 input
\usepackage[T1]{fontenc}    % use 8-bit T1 fonts
\usepackage{hyperref}       % hyperlinks
\usepackage{url}            % simple URL typesetting
\usepackage{booktabs}       % professional-quality tables
\usepackage{amsfonts}       % blackboard math symbols
\usepackage{nicefrac}       % compact symbols for 1/2, etc.
\usepackage{microtype}      % microtypography
\usepackage{apalike}
\usepackage{amssymb}


\title{Supporting derivations: Bayesian methods for efficient Reinforcement Learning in tabular problems}

% The \author macro works with any number of authors. There are two commands
% used to separate the names and addresses of multiple authors: \And and \AND.
%
% Using \And between authors leaves it to LaTeX to determine where to break the
% lines. Using \AND forces a line break at that point. So, if LaTeX puts 3 of 4
% authors names on the first line, and the last on the second line, try using
% \AND instead of \And before the third author name.

\author{}

\begin{document}

\maketitle



\section*{Moment matching across the Bellman equations}

The state and action-returns are respectively defined as:
\begin{align*}
\wb &\equiv \sum_{t = 1}^T \gamma^{~t - 1} r_t \big| \pi, \s_1 = \s, \mct, \mcr, ~ \text{ and } ~ \zb \equiv \sum_{t = 1}^T \gamma^{~t - 1} r_t \big| \pi, \s_1 = \s, \ac_1 = \ac, \mct, \mcr.
\end{align*}
They satisfy corresponding recursive relations:
\begin{align} \label{eq:recursivew}
\wb &=  \rsanb + \gamma \wnb, ~~ \ac \sim \pi, \ns \sim \mct\\
\zb &=  \rsanb + \gamma \znb, ~~ \ac' \sim \pi, \ns \sim \mct \label{eq:recursivez}
\end{align}
where equality is here taken to mean that the two sides are identically distributed. We can explicitly require that the first and second-order moments across these BEs are equal. In subsequent discussion, we use $\thetamct, \thetamcr, \thetamcwpi$ and $\thetamczpi$ to denote the parameters of $\mct, \mcr, \mcwpi$ and $\mczpi$ respectively.

\subsection*{First moment matching}

Taking expectations of both sides of \cref{eq:recursivew}, we obtain
\begin{equation}\label{muconsist}
\begin{aligned} 
\mathbb{E}_{w, \thetamcw}[\wb] &=  \mathbb{E}_{r, \thetamcr, \ns, \thetamct, \ac}[\rsanb] + \gamma  \mathbb{E}_{w, \thetamcw, \ns, \thetamct}[\wnb], \\
\mathbb{E}_{z, \thetamcz}[\zb] &=  \mathbb{E}_{r, \thetamcr, \ns, \thetamct}[\rsanb] + \gamma \mathbb{E}_{z, \thetamcz, \ns, \thetamct, \ac'}[\znb]
\end{aligned}
\end{equation}
where the expectations are taken over the posterior distributions of the subscript variables. We recognise as the familiar BEs in terms of expectations for $\V$ and $\Q$. These encode the requirement that, in expectation, the rewards and state/action-values should be consistent.

\subsection*{Second moment matching}

Taking variances of both sides of \cref{eq:recursivew} and \cref{eq:recursivez}, and using the laws of total variance and total covariance, we obtain similar consistency requirements for the variances:
\begin{equation}
\begin{aligned}
\Var_{w, \thetamcw}[\wb] &=  \Var_{r, \thetamcr, \ns, \thetamct, \ac}[\rsanb] + \gamma  \Var_{w, \thetamcw, \ns, \thetamct}[\wnb], \\
\Var_{z, \thetamcz}[\zb] &=  \Var_{r, \thetamcr, \ns, \thetamct}[\rsanb] + \gamma \Var_{z, \thetamcz, \ns, \thetamct, \ac'}[\znb]
\end{aligned}
\end{equation}
where variances are taken over the posterior distributions of the subscript variables. In subsequent discussion we assume a deterministic policy $\pi$ for simplicity, which implies that variances over $\ac$ and $\ac'$ are zero. It is straightforward to extend to the general case of a stochastic policy and we refrain from this for simplicity. Starting from the state-returns BE (\cref{eq:recursivew}) and using the law of total variance:
\begin{align}
\underbrace{\Var_{w, \thetamcw} \sqb{\wb}}_{\text{Total return unc.}} &= \underbrace{\Var_{\thetamcw} \sqb{\E_w\sqb{\wb | \thetamcw}}}_{\text{Epistemic return unc.}} + \underbrace{\E_{\thetamcw} \sqb{\Var_w\sqb{\wb | \thetamcw}}}_{\text{Aleatoric return unc.}}.
\end{align}

We expand the RHS of \cref{eq:recursivez} to obtain:
\begin{align}
\underbrace{\Var_{r, \thetamcr, w, \thetamcw, \ns, \thetamct, \ac} \sqb{\rsanb + \gamma \wnb}}_{\text{Next-return variance}} = &\underbrace{\Var_{r, \thetamcr, \ns, \thetamct}[\rsanb] }_{\text{Reward variance}}  \label{wvar} \\
+ &2\gamma \underbrace{\Cov_{r, \thetamcr, w, \thetamcw, \ns, \thetamct}[\rsanb, \wnb]}_{\text{Reward-return covariance}} \nonumber \\
+& \gamma^2 \underbrace{\Var_{w, \thetamcw, \ns, \thetamct}[\wnb]}_{\text{Next-return variance}}. \nonumber
\end{align}
The variances in \cref{wvar} contain both aleatoric and epistemic contributions, which we aim to separate by using the laws of total variance and total covariance (\cite{weiss}). The reward variance in \cref{wvar} can be expanded as:
\begin{align} \label{varrew}
\underbrace{\Var_{r, \thetamcr, \ns, \thetamct}[\rsanb]}_{\text{Reward variance}} = &\underbrace{\Var_{\ns, \thetamct}\sqb{ \E_{r, \thetamcr} \sqb{\rsanb \big| \ns, \thetamct}}}_{\substack{\text{Reward variance from transition unc.} \\ \text{aleatoric + epistemic}}}  \\ 
+ &\underbrace{\E_{\ns, \thetamct}\sqb{ \Var_{r, \thetamcr} \sqb{\rsanb \big| \ns, \thetamct}}}_{\substack{\text{Reward variance from reward unc.} \\ \text{aleatoric + epistemic}}}.\nonumber
\end{align}
Applying total variance to the first term in \cref{varrew}, we obtain
\begin{align} 
\underbrace{\Var_{\ns, \thetamct}\sqb{ \E_{r, \thetamcr} \sqb{\rsanb \big| \ns, \thetamct}}}_{\substack{\text{Reward variance from transition unc.} \\ \text{aleatoric + epistemic}}} =& \underbrace{\Var_{\thetamct} \sqb{ \E_{\ns, r, \thetamcr} \sqb{\rsanb \big| \thetamct}}}_{\substack{\text{Reward variance from transition unc.} \\ \text{purely epistemic}}} \\
+& \underbrace{\E_{\thetamct}\sqb{ \Var_{\ns} \sqb{ \E_{r, \thetamcr} \sqb{\rsanb \big| \ns, \thetamct}}}}_{\substack{\text{Reward variance from transition unc.} \\ \text{purely aleatoric}}}\nonumber
\end{align}
Similarly for the second term in \cref{varrew}:
\begin{align}\label{appeq:rerwvarrew}
\underbrace{\E_{\ns, \thetamct}\sqb{ \Var_{r, \thetamcr} \sqb{\rsanb \big| \ns, \thetamct}}}_{\substack{\text{Reward variance from reward unc.} \\ \text{aleatoric + epistemic}}} = &\underbrace{\E_{\ns, \thetamct}\sqb{ \Var_{\thetamcr} \sqb{ \E_r \sqb{\rsanb \big| \ns, \thetamct, \thetamcr}}}}_{\substack{\text{Reward variance from reward unc.} \\ \text{purely epistemic}}} \\ 
&+ \underbrace{\E_{\ns, \thetamct}\sqb{ \E_{\thetamcr} \sqb{ \Var_r \sqb{\rsanb \big| \ns, \thetamct, \thetamcr}}}}_{\substack{\text{Reward variance from reward unc.} \\ \text{purely aleatoric}}},\nonumber
\end{align}
with which we conclude the expansion of the reward variance term. We apply the same steps for the state-return variance in \cref{wvar}. By total variance:
\begin{align} \label{vvardecomp}
\underbrace{\Var_{w, \thetamcw, \ns, \thetamct}\sqb{\wnb}}_{\text{Next-step value variance}} =& \underbrace{\Var_{\ns, \thetamct}\sqb{\E_{w, \thetamcw} \sqb{\wnb | \ns, \thetamct}}}_{\substack{\text{State-return variance from transition unc.} \\ \text{aleatoric + epistemic}}} + \underbrace{\E_{\ns, \thetamct}\sqb{\Var_{w, \thetamcw} \sqb{\wnb | \ns, \thetamct}}}_{\substack{\text{State-return variance from state-return unc.} \\ \text{aleatoric + epistemic}}}.
\end{align}
Decomposing each of the two terms in \cref{vvardecomp} by total variance, we obtain
\begin{align*}
\underbrace{\Var_{\ns, \thetamct}\sqb{\E_{w, \thetamcw} \sqb{\wnb | \ns, \thetamct}}}_{\substack{\text{State-return variance from transition unc.} \\ \text{aleatoric + epistemic unc.}}} =& \underbrace{\Var_{\thetamct} \sqb{\E_{\ns, w, \thetamcw} \sqb{\wnb | \thetamct}}}_{\substack{\text{State-return variance from transition unc.} \\ \text{purely epistemic}}} \\
& + \underbrace{\E_{\thetamct}\sqb{ \Var_{\ns} \sqb{\E_{w, \thetamcw} \sqb{\wnb | \ns, \thetamct}}}}_{\substack{\text{State-return variance from transition unc.} \\ \text{purely aleatoric}}},
\end{align*}
for the first term. For the second term:
\begin{align*}
\underbrace{\E_{\ns, \thetamct}\sqb{\Var_{w, \thetamcw} \sqb{\wnb | \ns, \thetamct}}}_{\substack{\text{State-return variance from state-return unc.} \\ \text{aleatoric + epistemic}}} &=  \underbrace{\E_{\ns, \thetamct}\sqb{\Var_{\thetamcw} \sqb{ \E_w \sqb{\wnb | \ns, \thetamct}}}}_{\substack{\text{State-return variance from state-return unc.} \\ \text{purely epistemic}}} +\\
&+  \underbrace{\E_{\ns, \thetamct}\sqb{\E_{\thetamcw} \sqb{ \Var_w \sqb{\wnb | \ns, \thetamct}}}}_{\substack{\text{State-return variance from state-return unc.} \\ \text{purely aleatoric}}}.
\end{align*}
which concludes the decomposition of the value variance terms. For the reward-value covariance term, we use the law of total covariance to obtain
\begin{align}
\underbrace{\Cov_{r, \thetamcr, w, \thetamcw, \ns, \thetamct}[\rsanb, \wnb]}_{\text{Reward-return covariance}} =&~\underbrace{\Cov_{\ns, \thetamct}\sqb{\E_{r, \thetamcr}\sqb{\rsanb | \ns, \thetamct}, \E_{w, \thetamcw}\sqb{\wnb | \ns, \thetamct}}}_{\substack{\text{Reward-return covariance due to transition unc.} \\ \text{epistemic + aleatoric}}} + \nonumber \\
& + \cancel{\E_{\ns, \thetamct}\sqb{\Cov_{r, \thetamcr, w, \thetamcw}\sqb{\rsanb, \wnb | \ns, \thetamct}}}\label{appeq:covariance}
\end{align}
where the second term is 0 due to the conditional independence of $\rsanb$ and and $\wnb$ given $\ns$. Applying total covariance to \cref{appeq:covariance}:
\begin{align} \label{appeq:covrewval} 
&\underbrace{\Cov_{\ns, \thetamct}\sqb{\E_{r, \thetamcr}\sqb{\rsanb | \ns, \thetamct}, \E_{w, \thetamcw}\sqb{\wnb | \ns, \thetamct}}}_{\substack{\text{Reward-value covariance due to dynamics} \\ \text{epistemic + aleatoric}}}  = \underbrace{\Cov_{\thetamct}\sqb{\E_{\ns, r, \thetamcr}\sqb{\rsanb | \thetamct}, \E_{\ns, w, \thetamcw}\sqb{\wnb | \thetamct}}}_{\substack{\text{Reward-value covariance due to $\thetamct$ uncertainty} \\ \text{purely epistemic}}}  \\ 
&\qquad \qquad \qquad \qquad \qquad \qquad \qquad \qquad \qquad ~~~+\underbrace{\E_{\thetamct}\sqb{\Cov_{\ns}\sqb{\E_{r, \thetamcr}\sqb{\rsanb | \ns, \thetamct}, \E_{w, \thetamcw}\sqb{\wnb | \ns, \thetamct}}}}_{\substack{\text{Reward-value covariance due to dynamics stochasticity} \\ \text{purely aleatoric}}} \nonumber 
\end{align}
This concludes the decomposition of all terms of the RHS of \cref{wvar}. We consider the epistemic uncertainties only, and require that these are equal across \cref{wvar} to obtain:
\begin{align}
\underbrace{\Var_{\thetamcw} \sqb{\E_w\sqb{\wb | \thetamcw}}}_{\text{Epistemic state-return unc.}} =& \underbrace{\Var_{\thetamct} \sqb{ \E_{\ns, r, \thetamcr} \sqb{\rsanb \big| \thetamct}}}_{\substack{\text{Epistemic reward unc. from} \\ \text{dynamics unc.}}} + \underbrace{\E_{\ns, \thetamct}\sqb{ \Var_{\thetamcr} \sqb{ \E_r \sqb{\rsanb \big| \ns, \thetamct, \thetamcr}}}}_{\substack{\text{Epistemic rewards unc. from} \\ \text{rewards unc.}}} \nonumber + \\
& + 2 \gamma \underbrace{\Cov_{\thetamct}\sqb{\E_{\ns, r, \thetamcr}\sqb{\rsanb | \thetamct}, \E_{\ns, w, \thetamcw}\sqb{\wnb | \thetamct}}}_{\substack{\text{Epistemic reward and state-return covariance} \\ \text{from dynamics unc.}}} \nonumber\\
& + \gamma^2 \underbrace{\Var_{\thetamct} \sqb{\E_{\ns, w, \thetamcw} \sqb{\wnb | \thetamct}}}_{\substack{\text{Epistemic state-return unc. from} \\ \text{dynamics unc.}}}  + \gamma^2 \underbrace{\E_{\ns, \thetamct}\sqb{\Var_{\thetamcw} \sqb{ \E_w \sqb{\wnb | \ns, \thetamcw}}}}_{\substack{\text{Epistemic state-return unc. from} \\ \text{state-return unc.}}} \label{wepeq}
\end{align}
which concludes our derivation - note that wherever $\ac$ is seen as a free variable in the equation above, it is implied that $\ac = \pi(\s)$. Following the same argument, we obtain a consistency requirement for the action-returns:
\begin{align}
\underbrace{\Var_{\thetamcz} \sqb{\E_z\sqb{\zb | \thetamcz}}}_{\text{Epistemic action-return unc.}} =& \underbrace{\Var_{\thetamct} \sqb{ \E_{\ns, r, \thetamcr} \sqb{\rsanb \big| \thetamct}}}_{\substack{\text{Epistemic reward unc. from} \nonumber \\ \text{dynamics unc.}}} + \underbrace{\E_{\ns, \thetamct}\sqb{ \Var_{\thetamcr} \sqb{ \E_r \sqb{\rsanb \big| \ns, \thetamct, \thetamcr}}}}_{\substack{\text{Epistemic rewards unc. from} \\ \text{rewards unc.}}} + \\
& + 2 \gamma \underbrace{\Cov_{\thetamct}\sqb{\E_{\ns, r, \thetamcr}\sqb{\rsanb | \thetamct}, \E_{\ns, z, \thetamcz}\sqb{\znb | \thetamct}}}_{\substack{\text{Epistemic reward and action-return covariance} \\ \text{from dynamics unc.}}} \nonumber \\
& + \gamma^2 \underbrace{\Var_{\thetamct} \sqb{\E_{\ns, z, \thetamcz} \sqb{\znb | \thetamct}}}_{\substack{\text{Epistemic action-return unc. from} \\ \text{dynamics unc.}}}  +  \gamma^2 \underbrace{\E_{\ns, \thetamct}\sqb{\Var_{\thetamcz} \sqb{ \E_z \sqb{\znb | \ns, \thetamcz}}}}_{\substack{\text{Epistemic action-return unc. from} \\ \text{state-return unc.}}} \label{zepeq}
\end{align}
where again we have $\ac' = \pi(\ns)$ according to the deterministic policy. Note that all terms in the RHS of both \cref{wepeq} and \cref{zepeq} can be obtained either in closed form or by efficient MC integrals once \cref{muconsist} has been solved - except for the very last term in both cases.

We recognise the last term of the RHS is equal to the LHS term, smoothed by the predictive posterior over $\ns$. Therefore \cref{wepeq} and \cref{zepeq} are linear equations in the epistemic uncertainties of the returns which can be solved in $\mc{O}(|\mcs|^3)$ time.


\bibliographystyle{apalike}
\bibliography{references}

\end{document}